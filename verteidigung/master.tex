%%%%%%%%%%%%%%%%%%%%%%%%%%%%%%%%%%%%%%%%%%%%%%%%%%%%%%%%%%%%
%%  This Beamer template was created by Cameron Bracken.
%%  Anyone can freely use or modify it for any purpose
%%  without attribution.
%%
%%  Last Modified: January 9, 2009
%%

\documentclass[xcolor=x11names,compress]{beamer}

%% General document %%%%%%%%%%%%%%%%%%%%%%%%%%%%%%%%%%
\usepackage[utf8]{inputenc}
\usepackage[ngerman]{babel}
\usepackage{graphicx}
\usepackage{color}
\usepackage{tikz}
\usepackage{tabularx}
\usepackage[skins]{tcolorbox}
\usepackage{caption}
\usepackage{csquotes}
\usepackage{multicol,listings}
%\usepackage{subcaption}
\usepackage{multirow}
%%%%%%%%%%%%%%%%%%%%%%%%%%%%%%%%%%%%%%%%%%%%%%%%%%%%%%

%% Beamer Layout %%%%%%%%%%%%%%%%%%%%%%%%%%%%%%%%%%
\useoutertheme[subsection=false,infolines]{miniframes}
\useinnertheme{default}
\usefonttheme{serif}
\usepackage{palatino}

\setbeamertemplate{footline}[frame number]
%\setbeamertemplate{caption}[numbered]
\setbeamercolor{page number in head/foot}{fg=black}

\setbeamerfont{title like}{shape=\scshape}
\setbeamerfont{frametitle}{shape=\scshape}
\setbeamerfont{caption}{size=\tiny}
\setbeamerfont{caption name}{size=\tiny}

\setbeamercolor*{lower separation line head}{bg=DeepSkyBlue4} 
\setbeamercolor*{normal text}{fg=black,bg=white} 
\setbeamercolor*{alerted text}{fg=red} 
\setbeamercolor*{example text}{fg=black} 
\setbeamercolor*{structure}{fg=black} 
 
\setbeamercolor*{palette tertiary}{fg=black,bg=black!10} 
\setbeamercolor*{palette quaternary}{fg=black,bg=black!10} 

\renewcommand{\(}{\begin{columns}}
\renewcommand{\)}{\end{columns}}
\newcommand{\<}[1]{\begin{column}{#1}}
\renewcommand{\>}{\end{column}}

\beamertemplatenavigationsymbolsempty
\setcounter{tocdepth}{1}

\tcbset{enhanced}

% Set color for lstlisting
\definecolor{grey}{RGB}{105,105,105}

%%%%%%%%%%%%%%%%%%%%%%%%%%%%%%%%%%%%%%%%%%%%%%%%%%


\begin{document}

\begin{frame}
  \begin{figure}
    \begin{minipage}[c]{0.6\textwidth} 
    \tiny{Humboldt-Universität zu Berlin\\Mathematisch-Naturwissenschaftliche Fakultät\\Institut für Informatik\\Lehrstuhl Technische Informatik}
    \end{minipage}
    \hfill
    \begin{minipage}[c]{0.15\textwidth}
    \begin{figure}
      \includegraphics[width=\textwidth]{figures/HU_Logo}
    \end{figure}
    \end{minipage}
  \end{figure}

\title{\textbf{Measurements and Optimizations with Just-In-Time Code Generation on the OpenFlow Reference Implementation}}
\subtitle{Verteidigung der Bachelorarbeit}

\author{
  \vspace*{-1cm}
	\normalsize{\it Samuel Brack}\\
}
\date{\today}
\titlepage
\end{frame}

%-----------------------------------------------------------------------------%

\begin{frame}
  \frametitle{Agenda}
  \tableofcontents
\end{frame}

\section{\scshape Einführung}
\subsection{\scshape Paketklassifikation}
\begin{frame}
  \frametitle{\insertsubsection}
  % Regelmenge
  % Modellierung als Funktion mit P, R -> D
  Paketklassifikation modellierbar als Funktion \textit{M}:
  \begin{tcolorbox}[colback=blue!5!white,colframe=blue!75!black,title=Definition,drop fuzzy shadow]
  \begin{tabularx}{\textwidth}{XX}
    Eingabe&Paket \textit{P}, Regelmenge \textit{R}\\
    Ausgabe&Entscheidung \textit{D}
  \end{tabularx}
  \end{tcolorbox}
\end{frame}

\begin{frame}
  \begin{figure}
  \centering
  \includegraphics[width=0.8\textwidth]{figures/paketheader}
  \caption{Beispielhafter Aufbau eines Netzwerkpakets.}
  \end{figure}
  \pause
  \tiny
  \begin{table}
  \begin{tabularx}{\textwidth}{XXXXX|X}
    IP-Quelle&IP-Ziel&Transport\-protokoll&Quellport&Zielport&Aktion\\
    \hline
    8.8.8.8/32&10.0.0.0/8&UDP&53&*&\textsc{DROP}\\
    10.0.0.0/8&8.8.8.8/32&UDP&*&53&\textsc{DROP}\\
    141.20.33.123/32&10.0.0.0/8&*&*&*&\textsc{DROP}\\
    10.0.0.0/8&141.20.33.123/32&*&*&*&\textsc{DROP}\\
    141.20.33.0/24&10.0.0.0/8&TCP&80&*&\textsc{ACCEPT}\\
    10.0.0.0/8&141.20.33.0/24&TCP&*&80&\textsc{ACCEPT}\\
    10.0.0.0/8&0.0.0.0/0&*&*&*&\textsc{DROP}
  \end{tabularx}
  \caption{Beispiel für eine Regelmenge mit fünf Header-Feldern.}
  \end{table}
  \normalsize
\end{frame}

\subsection{\scshape OpenFlow}
\begin{frame}
  \frametitle{\insertsubsection}
  
\end{frame}

\section{\scshape Der Bitvector-Algorithmus}
\subsection{\scshape Allgemeine Idee}
\begin{frame}
  Paketfilter sind häufig als einfache Liste von Regeln implementiert.\\
  \begin{tcolorbox}[colback=red!5!white,colframe=red!75!black,title=Problem,drop fuzzy shadow]
  Lineare Suche ist langsam, da nur in $\mathcal O(n)$.
  \end{tcolorbox}
  \pause
  \begin{tcolorbox}[colback=yellow!5!white,colframe=yellow!75!black,title=Grundidee,drop fuzzy shadow]
  Binäre Suche möglich in $\mathcal O(log\ n)$.\\
  Divide and conquer über die einzelnen Header-Felder.
  \end{tcolorbox}
  \pause
  \begin{tcolorbox}[colback=blue!5!white,colframe=blue!75!black,title=Mögliche Lösung,drop fuzzy shadow]
  Der Bitvector-Algorithmus.
  \end{tcolorbox}
\end{frame}

\begin{frame}
  Grundlegende Idee:\\
  Geometrische Interpretation der Paketklassifikation.
  \pause
  \begin{table}
  \centering
  \begin{tabularx}{0.7\textwidth}{c|X|X}
  Regel&Quelladressen&Zieladressen\\
  \hline
  1&3 -- 11&4 -- 13\\
  2&1 -- 5&2 -- 5\\
  3&8 -- 13&0 -- 3\\
  \end{tabularx}
  \caption{Beispielhafter Regelsatz.}
  \end{table}
  Im Folgenden: Beim ersten Treffer wird abgebrochen und diese Regel \enquote{gewinnt}.
\end{frame}

\begin{frame}
  \begin{figure}
  \centering
  \includegraphics<1>[height=0.7\textheight]{figures/bitvector-L1}
  \includegraphics<2>[height=0.7\textheight]{figures/bitvector-L1-2}
  \includegraphics<3>[height=0.7\textheight]{figures/bitvector-L1-3}
  \includegraphics<4>[height=0.7\textheight]{figures/bitvector-L1-4}
  \includegraphics<5>[height=0.7\textheight]{figures/bitvector-L1-5}
  \includegraphics<6>[height=0.7\textheight]{figures/bitvector-L1-7}
  \caption{Geometrische Interpretation der Regeln aus voriger Tabelle.}
  \end{figure}
\end{frame}

\subsection{\scshape Lookup von Paketen}
\begin{frame}
  \frametitle{\insertsubsection}
  Pakete treffen ein und die jeweils passende Regel wird gesucht:
  \begin{table}
  \centering
  \begin{tabularx}{0.6\textwidth}{c|c|c}
  Paket&Quelladresse&Zieladresse\\
  \hline
  P\ 1&4&4\\
  P\ 2&14&7\\
  \end{tabularx}
  \caption{Beim Paketfilter eingetroffene Pakete.}
  \end{table}
\end{frame}

\begin{frame}
  \begin{figure}
  \centering
  \includegraphics<1>[height=0.7\textheight]{figures/bitvector-L1-4_6_8_10}
  \includegraphics<2>[height=0.7\textheight]{figures/bitvector-L1-4_6_9_11}
  \caption{Lookup von Paketen.}
  \end{figure}
\end{frame}

\begin{frame}
  \begin{figure}
  \centering
  \includegraphics[height=0.7\textheight]{figures/matching}
  \caption{Zusammenführen der Ergebnisse.}
  \end{figure}
\end{frame}

\section{\scshape Die JIT-Komponente}
\subsection{\scshape Motivation}
\begin{frame}
  \frametitle{\insertsubsection}
  Bei statischem Regelsatz erfolgt die binäre Suche stets im selben Array (pro Dimension).\\
  \begin{figure}
  \centering
  \includegraphics[height=5.5cm]{figures/matching_process}
  \caption{Matching von Paketen bei statischem Regelsatz.}
  \end{figure}
\end{frame}

\begin{frame}
  \begin{tcolorbox}[colback=yellow!5!white,colframe=yellow!75!black,title=Erinnerung,drop fuzzy shadow]
  Die Klassifikationsfunktion \textit{M} hängt vom Regelsatz \textit{R} und dem Paket \textit{P} ab.
  \end{tcolorbox}
  \pause
  \begin{tcolorbox}[colback=blue!5!white,colframe=blue!75!black,title=Idee,drop fuzzy shadow]
  Weitere Optimierung durch Vorausberechnung einer Funktion $M_R$.\\
  Implementierung eines Suchbaumes pro Dimension in nativem Code.
  \end{tcolorbox}
  \pause
  \begin{tcolorbox}[colback=red!5!white,colframe=red!75!black,title=Nachteil,drop fuzzy shadow]
  Bei Regel-Update erfolgt komplette Neuberechnung des Codes für den Suchbaum.
  \end{tcolorbox}
\end{frame}

\begin{frame}
  \begin{figure}
  \centering
  \includegraphics[width=0.3\textwidth]{figures/array}
  \vfill
  \includegraphics[height=0.2\textheight]{figures/arrow_down}
  \vfill
  \includegraphics[height=3cm]{figures/bv-tree-simple}
  \caption{Generierter Suchbaum aus dem Regelsatz.}
  \end{figure}
\end{frame}

\subsection{\scshape JIT-Lookup}
\begin{frame}
  \frametitle{\insertsubsection}
  \begin{figure}
  \centering
  \includegraphics<1>[height=0.7\textheight]{figures/match_in_tree-L1-2}
  \includegraphics<2>[height=0.7\textheight]{figures/match_in_tree-L1_3}
  \includegraphics<3>[height=0.7\textheight]{figures/match_in_tree-L1_4}
  \includegraphics<4>[height=0.7\textheight]{figures/match_in_tree-L1_5}
  \caption{Matching eines Pakets im Suchbaum.}
  \end{figure}
\end{frame}

\subsection{\scshape Generierter Code}
\begin{frame}
\begin{multicols}{3}
\lstinputlisting[
    language={[x86masm]Assembler},
    breaklines=true,
    numbers=left,
    texcl=true,
    xleftmargin=5.0ex,
    basicstyle=\tiny\ttfamily,
    keywordstyle=\bfseries\color{red},
    commentstyle=\itshape\color{grey},
    identifierstyle=\color{blue},
    morekeywords={retq,cmpq},
    numberstyle=\tiny
    ]
    {jit_listing.asm}
\end{multicols}
\end{frame}

\section{\scshape Auswertung}
\begin{frame}
\end{frame}

\section{\scshape Fazit und Ausblick}
\begin{frame}
\end{frame}

\appendix
\section{\scshape Backup}
\begin{frame}[noframenumbering]
Backupslide
\end{frame}

\end{document}
