%%%%%%%%%%%%%%%%%%%%%%%%%%%%%%%%%%%%%%%%%%%%%%%%%%%%%%%%%%%%
%%  This Beamer template was created by Cameron Bracken.
%%  Anyone can freely use or modify it for any purpose
%%  without attribution.
%%
%%  Last Modified: January 9, 2009
%%

\documentclass[xcolor=x11names,compress]{beamer}

%% General document %%%%%%%%%%%%%%%%%%%%%%%%%%%%%%%%%%
\usepackage[utf8]{inputenc}
\usepackage[ngerman]{babel}
\usepackage{graphicx}
\usepackage{color}
\usepackage{tikz}
\usepackage{tabularx}
\usepackage[compatibility=false]{caption}
%\usepackage{subcaption}
\usepackage{multirow}
%%%%%%%%%%%%%%%%%%%%%%%%%%%%%%%%%%%%%%%%%%%%%%%%%%%%%%

%% Beamer Layout %%%%%%%%%%%%%%%%%%%%%%%%%%%%%%%%%%
\useoutertheme[subsection=false,infolines]{miniframes}
\useinnertheme{default}
\usefonttheme{serif}
\usepackage{palatino}

\setbeamertemplate{footline}[frame number]
\setbeamertemplate{caption}[numbered]
\setbeamercolor{page number in head/foot}{fg=black}

\setbeamerfont{title like}{shape=\scshape}
\setbeamerfont{frametitle}{shape=\scshape}
\setbeamerfont{caption}{size=\tiny}
\setbeamerfont{caption name}{size=\tiny}

\setbeamercolor*{lower separation line head}{bg=DeepSkyBlue4} 
\setbeamercolor*{normal text}{fg=black,bg=white} 
\setbeamercolor*{alerted text}{fg=red} 
\setbeamercolor*{example text}{fg=black} 
\setbeamercolor*{structure}{fg=black} 
 
\setbeamercolor*{palette tertiary}{fg=black,bg=black!10} 
\setbeamercolor*{palette quaternary}{fg=black,bg=black!10} 

\renewcommand{\(}{\begin{columns}}
\renewcommand{\)}{\end{columns}}
\newcommand{\<}[1]{\begin{column}{#1}}
\renewcommand{\>}{\end{column}}

\beamertemplatenavigationsymbolsempty
\setcounter{tocdepth}{2}

%%%%%%%%%%%%%%%%%%%%%%%%%%%%%%%%%%%%%%%%%%%%%%%%%%


\begin{document}

\begin{frame}
  \begin{figure}
    \begin{minipage}[c]{0.6\textwidth} 
    \tiny{Humboldt-Universität zu Berlin\\Mathematisch-Naturwissenschaftliche Fakultät\\Institut für Informatik\\Lehrstuhl Technische Informatik}
    \end{minipage}
    \hfill
    \begin{minipage}[c]{0.15\textwidth}
    \begin{figure}
      \includegraphics[width=\textwidth]{figures/HU_Logo}
    \end{figure}
    \end{minipage}
  \end{figure}

\title{\textbf{Measurements and Optimizations with Just-In-Time Code Generation on the OpenFlow Reference Implementation}}
\subtitle{Verteidigung der Bachelorarbeit}

\author{
  \vspace*{-1cm}
	\normalsize{\it Samuel Brack}\\
}
\date{\today}
\titlepage
\end{frame}

%-----------------------------------------------------------------------------%

\begin{frame}
  \frametitle{Agenda}
  \tableofcontents
\end{frame}

\section{\scshape Einführende Begriffe}
\subsection{Paketklassifikation}
\begin{frame}
  \frametitle{Paketklassifikation}
  
\end{frame}
\subsection{OpenFlow}
\begin{frame}
  \frametitle{OpenFlow}
  
\end{frame}

\section{\scshape Der Bitvector-Algorithmus}
\begin{frame}
  Paketfilter sind häufig als Liste von Regeln implementiert.\\
  Problem: Lineare Suche ist langsam, da nur in $\mathcal O(n)$.\\
  \pause
  Theoretisch beste Suche möglich in $\mathcal O(log\ n)$.\\ %TODO: quelle?
  Lösung: Divide and conquer über die einzelnen Header-Felder.\\
  \vspace{\baselineskip}
  \centering{Der Bitvector-Algorithmus}
\end{frame}

\begin{frame}
  Grundlegende Idee:\\
  Geometrische Interpretation der Paketklassifikation.
  \pause
  \begin{table}
  \centering
  \begin{tabularx}{0.7\textwidth}{c|X|X}
  Regel&Quelladressen&Zieladressen\\
  \hline
  1&3 -- 11&4 -- 13\\
  2&1 -- 5&2 -- 5\\
  3&8 -- 13&0 -- 3\\
  \end{tabularx}
  \caption{Beispielhafter Regelsatz.}
  \label{table:bv_ruleset}
  \end{table}
\end{frame}

\begin{frame}
  \begin{figure}
  \centering
  \includegraphics[height=0.7\textheight]{figures/bitvector-L1-2}
  \caption{Geometrische Interpretation der Regeln aus Tabelle \ref{table:bv_ruleset}.}
  \label{fig:bv-normal}
  \end{figure}
\end{frame}

\section{\scshape Die JIT-Komponente}
\begin{frame}
  Bei statischem Regelsatz erfolgt die binäre Suche stets im selben Array (pro Dimension).\\
  \begin{figure}
  \centering
  \includegraphics[height=5.5cm]{figures/matching_process}
  \caption{Matching von Paketen bei statischem Regelsatz.}
  \label{fig:matching-normal}
  \end{figure}
\end{frame}

\begin{frame}
  Idee: Weitere Optimierung durch Vorausberechnung.\\
  Implementierung eines Suchbaumes pro Dimension in nativem Code.\\
  Bei Regel-Update erfolgt Neuberechnung des Suchbaumes.\\
  Lookup: Funktionsaufruf mit Paketdatum als Argument.
\end{frame}

\section{\scshape Auswertung}
\begin{frame}
\end{frame}

\section{\scshape Fazit und Ausblick}
\begin{frame}
\end{frame}

\end{document}
