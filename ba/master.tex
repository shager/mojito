\documentclass[a4paper,
		12pt,
		parskip=full,
		titlepage
		]{scrartcl}

\usepackage[T1]{fontenc}
\usepackage{graphicx}
\usepackage{geometry}
\usepackage{titlesec}
\usepackage[utf8]{inputenc}
\usepackage{color}
\usepackage{hyperref}
\usepackage{float}


\usepackage{paralist}
\usepackage[section]{placeins} %don't float to next section

\usepackage[lofdepth,lotdepth]{subfig}

%\usepackage{minted}

\hyphenpenalty=10000


%\geometry{a4paper,left=40mm,right=40mm, top=20mm, bottom=40mm}

\linespread{1.25}

\graphicspath{{abbildungen/}} 
\titleformat{\section}[block]{\sffamily\Large\bfseries\filcenter}{\thesection}{1em}{}
\pagestyle{empty}

\newcommand{\blankpage}{
\newpage
\thispagestyle{empty}
\mbox{}
\newpage
}

%opening
\title{Dokument}
\author{Samuel Brack}
\date{}

\begin{document}
\thispagestyle{empty}

\hspace{20cm}
\vspace{-3cm}

\begin{figure}[H] \hspace{11cm}
\includegraphics[width=3.2 cm]{HU_Logo}
\end{figure}
% \vspace{5cm}
\begin{center}
  % \vspace{0.5 cm}
  \huge{\bf Title} \\ % Hier fuegen Sie den Titel Ihrer Arbeit ein.
  \vspace{1cm}
  \LARGE  Bachelorarbeit \\ % Geben Sie anstelle der Punkte an, ob es sich um eine
                % Diplomarbeit, eine Masterarbeit oder eine Bachelorarbeit handelt.
  \vspace{1cm}
  \Large zur Erlangung des akademischen Grades \\
  Bachelor of Science (B. Sc.)\\ % Bitte tragen Sie hier anstelle der Punkte ein:
         % Diplominformatiker(in),
         % Bachelor of Arts (B. A.),
         % Bachelor of Science (B. Sc.),
         % Master of Education (M. Ed.) oder
         % Master of Science (M. Sc.).
  \vspace{1.5cm}
  {\large
    \bf{
      \scshape
      Humboldt-Universit\"at zu Berlin \\
      Mathematisch-Naturwissenschaftliche Fakult\"at II \\
      Institut f\"ur Informatik\\
    }
  } 
  % \normalfont
\enlargethispage{10\baselineskip}
\end{center}
\vspace {4 cm}% gegebenenfalls kleiner, falls der Titel der Arbeit sehr lang sein sollte
%{3.2 cm} bei Verwendung von scrreprt, gegebenenfalls kleiner, falls der Titel der Arbeit sehr lang sein sollte
{\large
  \begin{tabular}{llll}
    eingereicht von:    &  && \\ % Bitte Vor- und Nachnamen anstelle der Punkte eintragen.
    geboren am:         &  && \\
    in:                 &  && \\
    &&&\\
    Gutachter: &  && \\
              &  && \\% Bitte Namen der Gutachter(innen) anstelle der Punkte eintragen
                 % bei zwei männlichen Gutachtern kann das (innen) weggestrichen werden
    &&&\\
    eingereicht am:     &   \hspace{3cm} verteidigt am: &  \\ % Bitte lassen Sie
                                    % diese beiden Felder leer.
                                    % Loeschen Sie ggf. das letzte Feld, wenn
                                    % Sie Ihre Arbeit laut Pruefungsordnung nicht
                                    % verteidigen muessen.
  \end{tabular}
}

\newpage


\chapter{Statement of authorship}

\thispagestyle{empty}

%%%%%%%%%%%%%%%%%%%%%%%%%%%%%%%%%%%%%%%%%%%%%%%%%%%%%%%%%%%%%%%%%%%%%%%%%%%%%%%%%%%%%%%%%%%%%%%%%%%%
%% Selbststaendigkeitserklaerung
%%%%%%%%%%%%%%%%%%%%%%%%%%%%%%%%%%%%%%%%%%%%%%%%%%%%%%%%%%%%%%%%%%%%%%%%%%%%%%%%%%%%%%%%%%%%%%%%%%%%

{\parindent 0cm
%%%%%%%%%%%%%%%%%%%%%%%%%%english version%%%%%%%%%%%%%%%%%%%%%%%%%%%%%%
 
I declare that I completed this thesis on my own and that information which has been
directly or indirectly taken from other sources has been noted as such. Neither this
nor a similar work has been presented to an examination committee.

  \vspace{3\baselineskip}
 
  Berlin, \today \hspace{0.25\linewidth}\parbox{0.3\linewidth}{\dotfill}
}
\newpage

\begin{abstract}

\end{abstract}

\addcontentsline{toc}{section}{Contents}
\setcounter{page}{1}
\tableofcontents{}

\chapter{Introduction}
The growth of the internet constantly poses new challenges to the producers of network equipment.
Today's applications like Multimedia, Web and Voice over IP are dependant of a transport network (the internet) with
high data rates, low latency, soft realtime properties and quality of service mechanisms.
This catalogue of network parameters is already implemented in the internet protocol stack.

However, due to Moore's Law the growth of the actual data traffic can not be handled by pure protocol optimizaions, 
but must be engineered in faster hardware.

The critical points in the internet's architecture concerning performance are the stations who have to decide for each packet what to do.
These are mostly Firewalls, Level-2-Switches and Level-3-Routers.
In general, these machines inspect the header data of every incoming IP packet and forward it following a ruleset that has been implemented before.
These rulesets can be mostly static (e.g. in case of a small router) or very dynamic (e.g. in a large stateful firewall).

One main point in optimization therefore lies in the matching algorithms executing this process.
Many rulesets require five header fields for matching: Source IP address, destination IP address, transport protocol, source port and destination port.
The last two fields imply the usage of TCP or UDP as transport protocol.
But there can be other relevant header fields, too, for example VLAN tags, QoS information or MAC source and destination addresses.

Matching algorithms become slower the more header fields they have to match.
Additionally, some fields require rules that imply certain ranges or prefixes, e.g. IP adresses.
These constraints make it more difficult to implement simple matching algorithms and in order to match with high speed, more sophisticated
algorithms have to be used for these fields.

The best-known example for non trivial matchable fields are IP addresses, where rules are normally defined as IP prefixes.
Prefixes in IP addresses correspond to subnets, because in IP prefixes there's always a certain amount of fixed bits at the beginning of the address
and the bits after that are free choosable. %korrektes Englisch?

An instance for a prefix of an IPv4 address can look as follows:
0110 1011 0001...
This means that the first 12 bit of the address space are fixed and that the remaining 20 bits may be choosen freely.
Now, that rule has to match packets with addresses from 0110 1011 0001 0000 0000 0000 0000 0000 up to
0110 1011 0001 1111 1111 1111 1111 1111. %TODO: unschön? evtl umformulieren

%TODO: Openflow/SDN
Software Defined Networking is the idea to decouple the two tasks of network hardware. %quote: sdn-whitepaper 
They are separated in two logical units: The control and the data plane.
The control plane is used for configuring and updating the hardware, e.g. inserting new rules into a firewall.
The data plane handles the incoming traffic and acts accordingly to the objectives defined in the control plane.

One popular instance of a SDN is OpenFlow.
This project is intended to open a virtual network on real hardware for educational and research usecases.
A key feature of OpenFlow is the control channel system.
One controller has connections to all existing network hardware and can define their behaviour from a central position.
The hardware then acts accordingly and handles the traffic as required.

The main objective of this work is to implement a faster matching engine in the OpenFlow 1.3 reference implementation.


\pagebreak

\pagestyle{headings}

\newpage
\addcontentsline{toc}{section}{References}
\bibliographystyle{acm}
\bibliography{refs.bib,paper-build/conferences-crossref}

\end{document}
