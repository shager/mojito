\documentclass[a4paper,
		12pt,
		oneside,ngerman
		]{scrartcl}
\usepackage[german,ngerman]{babel}

\usepackage[T1]{fontenc}
\usepackage{graphicx}
\usepackage{geometry}
\usepackage[utf8]{inputenc}
\usepackage{float}
\usepackage{paralist}
\usepackage{cite}
\usepackage{url}
\usepackage{eurosym}

\linespread{1.25}

%opening
\title{Proposal zur Bachelorarbeit ``MOJITO: Measurements and Optimizations with Just-In-Time code generation on the Openflow reference implementation''}
\author{Brack Samuel}
\date{\today}

\begin{document}
\maketitle
OpenFlow ist eine Spezifikation für Switches, deren Implementierungen es Netzwerkadministatoren erlauben die Switches mit weit mehr Freiheitsgraden zu konfigurieren als bei herkömmlicher Hardware.
Die Weiterleitungstabelle wird dabei durch mehrere Flow-Tabellen ersetzt.
Diese können manuell editiert werden.

Dabei gibt es im Netzwerk einen Controller, welcher die Switches steuert und in jedem der Switches die Einträge der Flow-Tables modifiziert.
Die Flow-Tabellen sind je eine Menge von Flow-Einträgen, welche wiederum je ein Tupel aus einigen Headerdaten eines Pakets sind. 
Dadurch wird eine Zugehörigkeit von Paketen zu Flows definiert.
Zum Beispiel kann ein Flow alle Pakete beinhalten, die vom Host A gesendet werden und an UDP-Port B adressiert sind.
Diese Flow-Zuordnung kann auf mehreren Geräten im Netzwerk erfolgen, da der Kontrollrechner die Netzwerkgeräte über einen eigenen sicheren Kanal ansprechen und konfigurieren kann.

Die Arbeit basiert auf der Modifikation der Linux-Referenzimplementierung eines OpenFlow-Switches.
Es wird angestrebt die bestehenden Algorithmen zur Auswahl des korrekten Flow-Eintrags aus der Flow-Tabelle bei Ankunft eines Pakets zu ergänzen.

Verschiedene Ziele bieten sich in dieser Umgebung an:
\begin{enumerate}
    \item Implementierung einer bereits bekannten, noch nicht eingebauten Struktur zum schnellen Matching in der Tabelle (siehe auch das ATTI-Projekt); bisher existieren nur eine lineare Suche und eine Hash-Suche.
    \item Implementierung des Matchers als JIT-Funktion in x86\_64-Maschinencode, die als Eingabe das zu untersuchenden Paket erhält und dann den darauf passenden Flow-Eintrag möglichst schnell ermittelt.
\end{enumerate}

Dabei stellt (1) das Minimalziel der Arbeit dar, wobei (2) ein Spezialfall von (1) ist.

Das zweite vorgestellte Ziel soll den primären Fokus der Arbeit darstellen und bietet auch ausreichend Potential, um weiterführende interessante Fragestellungen zu bearbeiten.
Einige Beispiele hierfür sind:
\begin{enumerate}
    \item Anfallende Traffic-Muster sollen auch zur Optimierung der Flow-Tabelle genutzt werden, um im average case möglichst kurze Entscheidungszeiten zu erreichen.
    \item Effiziente Gestaltung der Einfüge- und Entferungsoperationen von Einträgen der Flow-Tabelle.
\end{enumerate}

Die Evaluierung der Implementierung erfolgt idealerweise mithilfe eines Trafficgenerators, der verschiedene Traffic-Szenarien aus der realen Welt nachbilden kann und eine hohe Durchsatzrate bietet.
\end{document}
